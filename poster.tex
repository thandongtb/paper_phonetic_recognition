
%% bare_conf.tex
%% V1.3
%% 2007/01/11
%% by Michael Shell
%% See:
%% http://www.michaelshell.org/
%% for current contact information.
%%
%% This is a skeleton file demonstrating the use of IEEEtran.cls
%% (requires IEEEtran.cls version 1.7 or later) with an IEEE conference paper.
%%
%% Support sites:
%% http://www.michaelshell.org/tex/ieeetran/
%% http://www.ctan.org/tex-archive/macros/latex/contrib/IEEEtran/
%% and
%% http://www.ieee.org/

%%*************************************************************************
%% Legal Notice:
%% This code is offered as-is without any warranty either expressed or
%% implied; without even the implied warranty of MERCHANTABILITY or
%% FITNESS FOR A PARTICULAR PURPOSE! 
%% User assumes all risk.
%% In no event shall IEEE or any contributor to this code be liable for
%% any damages or losses, including, but not limited to, incidental,
%% consequential, or any other damages, resulting from the use or misuse
%% of any information contained here.
%%
%% All comments are the opinions of their respective authors and are not
%% necessarily endorsed by the IEEE.
%%
%% This work is distributed under the LaTeX Project Public License (LPPL)
%% ( http://www.latex-project.org/ ) version 1.3, and may be freely used,
%% distributed and modified. A copy of the LPPL, version 1.3, is included
%% in the base LaTeX documentation of all distributions of LaTeX released
%% 2003/12/01 or later.
%% Retain all contribution notices and credits.
%% ** Modified files should be clearly indicated as such, including  **
%% ** renaming them and changing author support contact information. **
%%
%% File list of work: IEEEtran.cls, IEEEtran_HOWTO.pdf, bare_adv.tex,
%%                    bare_conf.tex, bare_jrnl.tex, bare_jrnl_compsoc.tex
%%*************************************************************************

% *** Authors should verify (and, if needed, correct) their LaTeX system  ***
% *** with the testflow diagnostic prior to trusting their LaTeX platform ***
% *** with production work. IEEE's font choices can trigger bugs that do  ***
% *** not appear when using other class files.                            ***
% The testflow support page is at:
% http://www.michaelshell.org/tex/testflow/



% Note that the a4paper option is mainly intended so that authors in
% countries using A4 can easily print to A4 and see how their papers will
% look in print - the typesetting of the document will not typically be
% affected with changes in paper size (but the bottom and side margins will).
% Use the testflow package mentioned above to verify correct handling of
% both paper sizes by the user's LaTeX system.
%
% Also note that the "draftcls" or "draftclsnofoot", not "draft", option
% should be used if it is desired that the figures are to be displayed in
% draft mode.
%
\documentclass[conference]{IEEEtran}
\usepackage{blindtext, graphicx}
\usepackage{hyperref}
\usepackage{caption}
\usepackage{amsmath}
\usepackage{amsfonts}
\usepackage{amssymb}
\usepackage{csquotes}
\usepackage[nocaptions]{vntex}
\usepackage{url}
\usepackage[table,xcdraw]{xcolor}
\newcommand\abs[1]{\left|#1\right|}
% Add the compsoc option for Computer Society conferences.
%
% If IEEEtran.cls has not been installed into the LaTeX system files,
% manually specify the path to it like:
% \documentclass[conference]{../sty/IEEEtran}





% Some very useful LaTeX packages include:
% (uncomment the ones you want to load)


% *** MISC UTILITY PACKAGES ***
%
%\usepackage{ifpdf}
% Heiko Oberdiek's ifpdf.sty is very useful if you need conditional
% compilation based on whether the output is pdf or dvi.
% usage:
% \ifpdf
%   % pdf code
% \else
%   % dvi code
% \fi
% The latest version of ifpdf.sty can be obtained from:
% http://www.ctan.org/tex-archive/macros/latex/contrib/oberdiek/
% Also, note that IEEEtran.cls V1.7 and later provides a builtin
% \ifCLASSINFOpdf conditional that works the same way.
% When switching from latex to pdflatex and vice-versa, the compiler may
% have to be run twice to clear warning/error messages.






% *** CITATION PACKAGES ***
%
%\usepackage{cite}
% cite.sty was written by Donald Arseneau
% V1.6 and later of IEEEtran pre-defines the format of the cite.sty package
% \cite{} output to follow that of IEEE. Loading the cite package will
% result in citation numbers being automatically sorted and properly
% "compressed/ranged". e.g., [1], [9], [2], [7], [5], [6] without using
% cite.sty will become [1], [2], [5]--[7], [9] using cite.sty. cite.sty's
% \cite will automatically add leading space, if needed. Use cite.sty's
% noadjust option (cite.sty V3.8 and later) if you want to turn this off.
% cite.sty is already installed on most LaTeX systems. Be sure and use
% version 4.0 (2003-05-27) and later if using hyperref.sty. cite.sty does
% not currently provide for hyperlinked citations.
% The latest version can be obtained at:
% http://www.ctan.org/tex-archive/macros/latex/contrib/cite/
% The documentation is contained in the cite.sty file itself.






% *** GRAPHICS RELATED PACKAGES ***
%
\ifCLASSINFOpdf
  % \usepackage[pdftex]{graphicx}
  % declare the path(s) where your graphic files are
  % \graphicspath{{../pdf/}{../jpeg/}}
  % and their extensions so you won't have to specify these with
  % every instance of \includegraphics
  % \DeclareGraphicsExtensions{.pdf,.jpeg,.png}
\else
  % or other class option (dvipsone, dvipdf, if not using dvips). graphicx
  % will default to the driver specified in the system graphics.cfg if no
  % driver is specified.
  % \usepackage[dvips]{graphicx}
  % declare the path(s) where your graphic files are
  % \graphicspath{{../eps/}}
  % and their extensions so you won't have to specify these with
  % every instance of \includegraphics
  % \DeclareGraphicsExtensions{.eps}
\fi
% graphicx was written by David Carlisle and Sebastian Rahtz. It is
% required if you want graphics, photos, etc. graphicx.sty is already
% installed on most LaTeX systems. The latest version and documentation can
% be obtained at: 
% http://www.ctan.org/tex-archive/macros/latex/required/graphics/
% Another good source of documentation is "Using Imported Graphics in
% LaTeX2e" by Keith Reckdahl which can be found as epslatex.ps or
% epslatex.pdf at: http://www.ctan.org/tex-archive/info/
%
% latex, and pdflatex in dvi mode, support graphics in encapsulated
% postscript (.eps) format. pdflatex in pdf mode supports graphics
% in .pdf, .jpeg, .png and .mps (metapost) formats. Users should ensure
% that all non-photo figures use a vector format (.eps, .pdf, .mps) and
% not a bitmapped formats (.jpeg, .png). IEEE frowns on bitmapped formats
% which can result in "jaggedy"/blurry rendering of lines and letters as
% well as large increases in file sizes.
%
% You can find documentation about the pdfTeX application at:
% http://www.tug.org/applications/pdftex





% *** MATH PACKAGES ***
%
%\usepackage[cmex10]{amsmath}
% A popular package from the American Mathematical Society that provides
% many useful and powerful commands for dealing with mathematics. If using
% it, be sure to load this package with the cmex10 option to ensure that
% only type 1 fonts will utilized at all point sizes. Without this option,
% it is possible that some math symbols, particularly those within
% footnotes, will be rendered in bitmap form which will result in a
% document that can not be IEEE Xplore compliant!
%
% Also, note that the amsmath package sets \interdisplaylinepenalty to 10000
% thus preventing page breaks from occurring within multiline equations. Use:
%\interdisplaylinepenalty=2500
% after loading amsmath to restore such page breaks as IEEEtran.cls normally
% does. amsmath.sty is already installed on most LaTeX systems. The latest
% version and documentation can be obtained at:
% http://www.ctan.org/tex-archive/macros/latex/required/amslatex/math/





% *** SPECIALIZED LIST PACKAGES ***
%
%\usepackage{algorithmic}
% algorithmic.sty was written by Peter Williams and Rogerio Brito.
% This package provides an algorithmic environment fo describing algorithms.
% You can use the algorithmic environment in-text or within a figure
% environment to provide for a floating algorithm. Do NOT use the algorithm
% floating environment provided by algorithm.sty (by the same authors) or
% algorithm2e.sty (by Christophe Fiorio) as IEEE does not use dedicated
% algorithm float types and packages that provide these will not provide
% correct IEEE style captions. The latest version and documentation of
% algorithmic.sty can be obtained at:
% http://www.ctan.org/tex-archive/macros/latex/contrib/algorithms/
% There is also a support site at:
% http://algorithms.berlios.de/index.html
% Also of interest may be the (relatively newer and more customizable)
% algorithmicx.sty package by Szasz Janos:
% http://www.ctan.org/tex-archive/macros/latex/contrib/algorithmicx/




% *** ALIGNMENT PACKAGES ***
%
%\usepackage{array}
% Frank Mittelbach's and David Carlisle's array.sty patches and improves
% the standard LaTeX2e array and tabular environments to provide better
% appearance and additional user controls. As the default LaTeX2e table
% generation code is lacking to the point of almost being broken with
% respect to the quality of the end results, all users are strongly
% advised to use an enhanced (at the very least that provided by array.sty)
% set of table tools. array.sty is already installed on most systems. The
% latest version and documentation can be obtained at:
% http://www.ctan.org/tex-archive/macros/latex/required/tools/


%\usepackage{mdwmath}
%\usepackage{mdwtab}
% Also highly recommended is Mark Wooding's extremely powerful MDW tools,
% especially mdwmath.sty and mdwtab.sty which are used to format equations
% and tables, respectively. The MDWtools set is already installed on most
% LaTeX systems. The lastest version and documentation is available at:
% http://www.ctan.org/tex-archive/macros/latex/contrib/mdwtools/


% IEEEtran contains the IEEEeqnarray family of commands that can be used to
% generate multiline equations as well as matrices, tables, etc., of high
% quality.


%\usepackage{eqparbox}
% Also of notable interest is Scott Pakin's eqparbox package for creating
% (automatically sized) equal width boxes - aka "natural width parboxes".
% Available at:
% http://www.ctan.org/tex-archive/macros/latex/contrib/eqparbox/





% *** SUBFIGURE PACKAGES ***
%\usepackage[tight,footnotesize]{subfigure}
% subfigure.sty was written by Steven Douglas Cochran. This package makes it
% easy to put subfigures in your figures. e.g., "Figure 1a and 1b". For IEEE
% work, it is a good idea to load it with the tight package option to reduce
% the amount of white space around the subfigures. subfigure.sty is already
% installed on most LaTeX systems. The latest version and documentation can
% be obtained at:
% http://www.ctan.org/tex-archive/obsolete/macros/latex/contrib/subfigure/
% subfigure.sty has been superceeded by subfig.sty.



%\usepackage[caption=false]{caption}
%\usepackage[font=footnotesize]{subfig}
% subfig.sty, also written by Steven Douglas Cochran, is the modern
% replacement for subfigure.sty. However, subfig.sty requires and
% automatically loads Axel Sommerfeldt's caption.sty which will override
% IEEEtran.cls handling of captions and this will result in nonIEEE style
% figure/table captions. To prevent this problem, be sure and preload
% caption.sty with its "caption=false" package option. This is will preserve
% IEEEtran.cls handing of captions. Version 1.3 (2005/06/28) and later 
% (recommended due to many improvements over 1.2) of subfig.sty supports
% the caption=false option directly:
%\usepackage[caption=false,font=footnotesize]{subfig}
%
% The latest version and documentation can be obtained at:
% http://www.ctan.org/tex-archive/macros/latex/contrib/subfig/
% The latest version and documentation of caption.sty can be obtained at:
% http://www.ctan.org/tex-archive/macros/latex/contrib/caption/




% *** FLOAT PACKAGES ***
%
%\usepackage{fixltx2e}
% fixltx2e, the successor to the earlier fix2col.sty, was written by
% Frank Mittelbach and David Carlisle. This package corrects a few problems
% in the LaTeX2e kernel, the most notable of which is that in current
% LaTeX2e releases, the ordering of single and double column floats is not
% guaranteed to be preserved. Thus, an unpatched LaTeX2e can allow a
% single column figure to be placed prior to an earlier double column
% figure. The latest version and documentation can be found at:
% http://www.ctan.org/tex-archive/macros/latex/base/



%\usepackage{stfloats}
% stfloats.sty was written by Sigitas Tolusis. This package gives LaTeX2e
% the ability to do double column floats at the bottom of the page as well
% as the top. (e.g., "\begin{figure*}[!b]" is not normally possible in
% LaTeX2e). It also provides a command:
%\fnbelowfloat
% to enable the placement of footnotes below bottom floats (the standard
% LaTeX2e kernel puts them above bottom floats). This is an invasive package
% which rewrites many portions of the LaTeX2e float routines. It may not work
% with other packages that modify the LaTeX2e float routines. The latest
% version and documentation can be obtained at:
% http://www.ctan.org/tex-archive/macros/latex/contrib/sttools/
% Documentation is contained in the stfloats.sty comments as well as in the
% presfull.pdf file. Do not use the stfloats baselinefloat ability as IEEE
% does not allow \baselineskip to stretch. Authors submitting work to the
% IEEE should note that IEEE rarely uses double column equations and
% that authors should try to avoid such use. Do not be tempted to use the
% cuted.sty or midfloat.sty packages (also by Sigitas Tolusis) as IEEE does
% not format its papers in such ways.





% *** PDF, URL AND HYPERLINK PACKAGES ***
%
%\usepackage{url}
% url.sty was written by Donald Arseneau. It provides better support for
% handling and breaking URLs. url.sty is already installed on most LaTeX
% systems. The latest version can be obtained at:
% http://www.ctan.org/tex-archive/macros/latex/contrib/misc/
% Read the url.sty source comments for usage information. Basically,
% \url{my_url_here}.





% *** Do not adjust lengths that control margins, column widths, etc. ***
% *** Do not use packages that alter fonts (such as pslatex).         ***
% There should be no need to do such things with IEEEtran.cls V1.6 and later.
% (Unless specifically asked to do so by the journal or conference you plan
% to submit to, of course. )


% correct bad hyphenation here
\hyphenation{op-tical net-works semi-conduc-tor}


\begin{document}
%
% paper title
% can use linebreaks \\ within to get better formatting as desired
\title{Improvement the phonetic recognition with sequence-length MFCC features and deep bidirectional LSTM}


% author names and affiliations
% use a multiple column layout for up to three different
% affiliations
\author{\IEEEauthorblockN{Toan Pham Van}
\IEEEauthorblockA{\textit{Framgia Inc R\&D Group}\\
\fontfamily{qcr}\selectfont{pham.van.toan@framgia.com}}
\and
\IEEEauthorblockN{Hau Nguyen Thanh}
\IEEEauthorblockA{\textit{Framgia Inc R\&D Group}}
\fontfamily{qcr}\selectfont{nguyen.thanh.hau@framgia.com}
\and
\IEEEauthorblockN{Thanh Ta Minh}
\IEEEauthorblockA{\textit{Le Quy Don Technical University}}
\fontfamily{qcr}\selectfont{ta.minh.thanh@framgia.com}

}

% conference papers do not typically use \thanks and this command
% is locked out in conference mode. If really needed, such as for
% the acknowledgment of grants, issue a \IEEEoverridecommandlockouts
% after \documentclass

% for over three affiliations, or if they all won't fit within the width
% of the page, use this alternative format:
% 
%\author{\IEEEauthorblockN{Michael Shell\IEEEauthorrefmark{1},
%Homer Simpson\IEEEauthorrefmark{2},
%James Kirk\IEEEauthorrefmark{3}, 
%Montgomery Scott\IEEEauthorrefmark{3} and
%Eldon Tyrell\IEEEauthorrefmark{4}}
%\IEEEauthorblockA{\IEEEauthorrefmark{1}School of Electrical and Computer Engineering\\
%Georgia Institute of Technology,
%Atlanta, Georgia 30332--0250\\ Email: see http://www.michaelshell.org/contact.html}
%\IEEEauthorblockA{\IEEEauthorrefmark{2}Twentieth Century Fox, Springfield, USA\\
%Email: homer@thesimpsons.com}
%\IEEEauthorblockA{\IEEEauthorrefmark{3}Starfleet Academy, San Francisco, California 96678-2391\\
%Telephone: (800) 555--1212, Fax: (888) 555--1212}
%\IEEEauthorblockA{\IEEEauthorrefmark{4}Tyrell Inc., 123 Replicant Street, Los Angeles, California 90210--4321}}




% use for special paper notices
%\IEEEspecialpapernotice{(Invited Paper)}




% make the title area
\maketitle


\begin{abstract}
%\boldmath
Phonetic recognition is an meaningful problem in the many fields of speech analysis. This applications can mention as  dialect identification [1], mispronunciation detection [2], spoken document retrieval [3] etc. Some main approaches to solving this problem are improve the feature selection on input speech [4], apply some deep learning techniques [5][6][7] or combination both of them [8]. With the sequence data as the phonetics, the architecture based on recurrent neural network is an appropriate approach [9]. It is more powerful when combined with the improvement of features selection in input data. In our approach, we combining the Mel Frequency Cepstral Coefficients (MFCC) method with sequence-length for present the acoustic features of speech and using some RNN models to phonetic classification. All the experiments are implement on the Texas Instruments Massachusetts Institute of Technology (TIMIT) [10] phone recognition dataset. Especially, our data processing and features selention method give consistently better results with other researches when applied the same neural network model. Currently, we achieved the lowest error test rate - 15.24\% by using Bidirectional LSTM, which is currently the best result in TIMIT dataset and reduction of about 1.3\% over the last best result [5][6].

 
\end{abstract}
% IEEEtran.cls defaults to using nonbold math in the Abstract.
% This preserves the distinction between vectors and scalars. However,
% if the journal you are submitting to favors bold math in the abstract,
% then you can use LaTeX's standard command \boldmath at the very start
% of the abstract to achieve this. Many IEEE journals frown on math
% in the abstract anyway.

% Note that keywords are not normally used for peerreview papers.
\begin{IEEEkeywords}
Phonetic Recognition, MFCC features, sequence-length, bidirectional LSTM, TIMIT
\end{IEEEkeywords}






% For peer review papers, you can put extra information on the cover
% page as needed:
% \ifCLASSOPTIONpeerreview
% \begin{center} \bfseries EDICS Category: 3-BBND \end{center}
% \fi
%
% For peerreview papers, this IEEEtran command inserts a page break and
% creates the second title. It will be ignored for other modes.
\IEEEpeerreviewmaketitle



\section{Introduction}

In a general machine learning application, speech recognition technology is one of the most typical application recently. In speech recognition technology, given a sequence of acoustic observations, this technology decodes the corresponding sequence of words or phonemes. From that, we can use it for helping language learner in pronunciation. The typical neural network model is used for speech recognition system is recurrent neural network (RNN), an effective model in sequence-to-sequence problem.\\
In this paper, we introduce recurrent neural network model and some variants, along with some techniques to improve the accuracy for phonetic classification problem such as: sequence length, feature scaling, deep long-short-term memory (deep LSTM), bidirectional LSTM. With these techniques, the phonetic classification problem is greatly improved compared to the original model.\\
We evaluate the effectiveness of models using TIMIT dataset. The original data were converted to Mel Frequency Cepstral Coefficient (MFCC) features. MFCCs features were said to have better results in speech recognition problem. In each generated output of sequence, we use the results of previous and next steps by using bidirectional LSTM. In others, when applying feature scaling for input data, the training process will be more faster, and it gets better results. We get achieved 13.5\% PER, the best result in TIMIT dataset until now. 

% needed in second column of first page if using \IEEEpubid
%\IEEEpubidadjcol

% An example of a floating figure using the graphicx package.
% Note that \label must occur AFTER (or within) \caption.
% For figures, \caption should occur after the \includegraphics.
% Note that IEEEtran v1.7 and later has special internal code that
% is designed to preserve the operation of \label within \caption
% even when the captionsoff option is in effect. However, because
% of issues like this, it may be the safest practice to put all your
% \label just after \caption rather than within \caption{}.
%
% Reminder: the "draftcls" or "draftclsnofoot", not "draft", class
% option should be used if it is desired that the figures are to be
% displayed while in draft mode.
%
%\begin{figure}[!t]
%\centering
%\includegraphics[width=2.5in]{myfigure}
% where an .eps filename suffix will be assumed under latex, 
% and a .pdf suffix will be assumed for pdflatex; or what has been declared
% via \DeclareGraphicsExtensions.
%\caption{Simulation Results}
%\label{fig_sim}
%\end{figure}

% Note that IEEE typically puts floats only at the top, even when this
% results in a large percentage of a column being occupied by floats.


% An example of a double column floating figure using two subfigures.
% (The subfig.sty package must be loaded for this to work.)
% The subfigure \label commands are set within each subfloat command, the
% \label for the overall figure must come after \caption.
% \hfil must be used as a separator to get equal spacing.
% The subfigure.sty package works much the same way, except \subfigure is
% used instead of \subfloat.
%
%\begin{figure*}[!t]
%\centerline{\subfloat[Case I]\includegraphics[width=2.5in]{subfigcase1}%
%\label{fig_first_case}}
%\hfil
%\subfloat[Case II]{\includegraphics[width=2.5in]{subfigcase2}%
%\label{fig_second_case}}}
%\caption{Simulation results}
%\label{fig_sim}
%\end{figure*}
%
% Note that often IEEE papers with subfigures do not employ subfigure
% captions (using the optional argument to \subfloat), but instead will
% reference/describe all of them (a), (b), etc., within the main caption.


% An example of a floating table. Note that, for IEEE style tables, the 
% \caption command should come BEFORE the table. Table text will default to
% \footnotesize as IEEE normally uses this smaller font for tables.
% The \label must come after \caption as always.
%
%\begin{table}[!t]
%% increase table row spacing, adjust to taste
%\renewcommand{\arraystretch}{1.3}
% if using array.sty, it might be a good idea to tweak the value of
% \extrarowheight as needed to properly center the text within the cells
%\caption{An Example of a Table}
%\label{table_example}
%\centering
%% Some packages, such as MDW tools, offer better commands for making tables
%% than the plain LaTeX2e tabular which is used here.
%\begin{tabular}{|c||c|}
%\hline
%One & Two\\
%\hline
%Three & Four\\
%\hline
%\end{tabular}
%\end{table}


% Note that IEEE does not put floats in the very first column - or typically
% anywhere on the first page for that matter. Also, in-text middle ("here")
% positioning is not used. Most IEEE journals use top floats exclusively.
% Note that, LaTeX2e, unlike IEEE journals, places footnotes above bottom
% floats. This can be corrected via the \fnbelowfloat command of the
% stfloats package.


\section{Related works}

\subsection{Baseline}

A simple approach to solve this problem is using phoneme-based recognition and identifying pronunciation errors in the input speech of non-native speakers. But in actually, the accuracy for detection words is much higher than phonetic detection even for native speakers. It also mean that we can not directly apply  \textbf{Automatic Speech Recognition} system for mispronunciation detection. Instead we add to the ASR system a pronunciation model with possible faulty pronunciation variations are used to recognize the most likely phone sequences when it knew previous phones. Finally, the mispronunciation detection system is worked by executing the forced-alignment of ASR with extended pronunciation recognizer based on possible phonetic confusions. The figure below is a simulation of workflow in this system:

\begin{figure}[h]
\includegraphics[scale=0.6]{baseline.jpg}
\centering
\caption{\textit{ASR-based system to detect and diagnose L2 learners mispronunciation}}
\end{figure}

\subsection{Feature Extraction}
As discussed above, phone-level mispronunciation detection can detect mispronunciations in units of phones, words or sentences. Firstly, the speaker's speech samples are first converted to certain types of features such as Linear predictive cepstral coefficients (\textbf{LPCC}), Mel-frequency cepstral coefficients (\textbf{MFCC}), Power spectral analysis (\textbf{FFT}) or Mel scale cepstral analysis (\textbf{MEL}) etc. These features are used as the input of classifier. These allow the system to classify the mispronunciation by types. In our works, we have used 26 \textbf{MFCC} features as input features.

\subsection{Dataset}

In our works for phonetic classification and detection, we used the dataset named \textbf{Japanese Phonetic Database - JPD} [1]. The JPD provides IPA phonetic transcriptions that accurately indicate how Japanese names and words are pronounced in actual speechs, as well as accent codes, for each entry. It includes detail descriptions of 130,000 entries in Japanese.

The advantage of this dataset are: it includes the pitch accent position for each phonetic and it is so meaningful for many speech processing tasks. An example of phonetics collection in \textbf{JPD} is shown below.

\begin{figure}[h]
\includegraphics[scale=0.37]{dataset.png}
\centering
\caption{\textit{JPD phonetics dataset with pitch accent}}
\end{figure}

\subsection{Phonetic classification methods}

Deep learning is a technique used a lot recently. There are  several methods of deep learning that we use there. Recurrent Neural Network (RNN) is a class of artificial neural network where connections between units form a directed graph along a sequence. This allows exhibitting dynamic temporal behavior for a time sequence. We use RNN with 3 layers of LSTM cell. Another deep neural network class we use there is Convolutional Neural Network (CNN), it also find the connectivity between features and help us to improve the accuracy. The architecture of CNN we use there is combination of convolutional layers, maxpool and fully connected layers. Finally we use both RNN and CNN in a combination called deepspeech, they help us increasing the accuracy in this works.

\subsection{Result}

We tried some models of deep learning architecture mentioned above to training the extended pronunciation recognizer. Following our previous works, two metrics false acceptance rate \textbf{(FAR)} and false rejection rate \textbf{(FRR)} are used to measure the system performance. The comparison of our Neural Network accuracy with some algorithms is shown in \textit{\textbf{Table 1}}

\begin{table}[]
\centering
\caption{\textit{Japanese phonetic classifier performance of four ANN architectures}}
\label{my-label}
\begin{tabular}{|l|l|l|l|l|}
\hline
                                                           & {\color[HTML]{333333} \textit{\textbf{CNN 1D}}} & {\color[HTML]{333333} \textit{\textbf{Normal RNN}}} & {\color[HTML]{333333} \textit{\textbf{LSTM}}} & {\color[HTML]{333333} \textit{\textbf{CNN + LSTM}}} 
                                                           \\ \hline
{\color[HTML]{333333} \textit{\textbf{FAR}}} & 21.92                                       & 17.21 & 
15.93 & 
11.73                                                  \\ \hline
{\color[HTML]{333333} \textit{\textbf{FRR}}} & 11.29                                       & 9.32 & 
9.02                                       & 6.23                                               \\ \hline
\end{tabular}
\end{table}

\section{Methods}

\subsection{Recurrent Neural Network}
Recurrent Neural Networks (RNNs) are popular models that have shown great promise in many Natural Language Processing, Artificial Speech Recognition, Time Series, Sequence-to-sequence tasks. The idea behind RNNs is to make use of sequential information. Here is what a typical RNN look like:
\begin{figure}[h]
\includegraphics[scale=0.5]{rnn.png}
\centering
\caption{\textit{Recurrent Neural Network}}
\end{figure}
\begin{align*}
	&h_t = Ux_t + Wh_{t-1}\\
	&o_t = Vh_t
\end{align*}


\subsection{Long Short Term Memory}
In theory RNNs can make use of information in arbitrarily long sequences, but in practice they are limited to looking back only a few steps. From that, LSTM cell have been designed to get around this problem. Long Short Term Memory Network (LSTMs) are a special kind of RNN, capable of learning long-term dependencies.

\begin{figure}[h]
\includegraphics[scale=0.5]{lstm.png}
\centering
\caption{\textit{Long Short Term Memory Cell}}
\end{figure}

\begin{align*}
     &i_t = \sigma (W_{xi}x_t + W_{hi}h_{t-1} + W_{ci}c_{t-1} + b_i) \\
     &f_t = \sigma (W_{xf}x_t + W_{hf}h_{t-1} + W_{cf}c_{t-1} + b_f) \\
     &c_t = f_tc_{t-1} + i_t tanh(W_{xc}x_t + W_{hc}h_{t-1} + b_c) \\
     &o_t = \sigma (W_{xo}x_t + W_{ho}h_{t-1} + W_{co}c_{t-1} + b_o) \\
     &h_t = o_t tanh(c_t)
\end{align*} 

where $\sigma$ is the logistic sigmoid function, and $i$, $f$, $o$ and $c$ are respectively the \textit{input gate}, \textit{forget gate}, \textit{output gate} and \textit{cell memory} vectors.

\subsection{Bidirectional Recurrent Neural Networks}
Bidirectional Recurrent Neural Networks (BRNNs) are based on the idea that the output at time t may not only depend on the previous elements in the sequence, but also future elements. Bidirectional RNNs are quite simple. They are just two RNNs stacked on top of each other. The output is then computed based on the hidden state of both RNNs.

\begin{figure}[h]
\includegraphics[scale=0.4]{brnn.png}
\centering
\caption{\textit{Bidirectional Recurrent Neural Network}}
\end{figure}

%As illustrated in Figure 5, a BRNN computes the forward hidden sequence h, the backward hidden sequence hand the output sequence y by iterating the backward layer from t = T to 1, the forward layer from t = 1 to T and then updating the output layer


\subsection{Deep Bidirectional Recurrent Neural Networks}
Deep (Bidirectional) RNNs are similar to Bidirectional RNNs, only that we now have multiple layers per time step. In practice this gives us a higher learning capacity (but we also need a lot of training data).
<picture and explaination>

\subsection{Other techniques}
\subsubsection{Sequence length}

In this paper, we use the input sequence is a 2D array representing each utterance of the sentence. Where each row is a feature vector with 26 MFCC values. Number of columns is the number of feature vectors of the sentence which have the longest MFCC features, called "max length". If the sentence has the number of features less than "max length", vectors with a value of 0 will be added to fit with "max length". This will allow us to use Tensorflow in training. Also, with the use of sequence length, we will ignore the dependence between sentences when we connect the sentences together, obviously for greater efficiency because sentences completely independent.
\subsubsection{Feature Scaling}
Feature scaling is a method used to standardize the range of independent variables or features of data. In data processing, it is also known as data normalization/standardization and is generally performed during the data preprocessing step. By using feature scaling with input data before training, we get the better results and faster training. The technique we used there is standardization, first it subtracts the mean value (so standardized values always have a zero mean), and then it divides by the variance so that the resulting distribution has unit variance. In others, standardization is much less affected by outliers.
\begin{equation}
	x' = \frac{x - \bar{x}}{\sigma}
\end{equation}
Where ${\displaystyle x}$ is the original feature vector, ${\displaystyle {\bar {x}}}$ is the mean of that feature vector, and ${\displaystyle \sigma }$ is its standard deviation.

\section{Experiments}
Phoneme recognition experiments were performed on the TIMIT corpus. TIMIT contains a total of 6300 sentences, 10 sentences spoken by each of 630 speakers from 8 major dialect regions of the United States.



\section{Conclusion and future works}
We have used the combination of deep, bidirectional Long Short-term Memory RNNs with end-to-end training gives state-of-the-art results in phoneme recognition on the TIMIT database. An our plan is extend the system to large vocabulary speech recognition. Another plan would be use some another techniques in deep learning such as convolutional neural networks (CNNs), gated recurrent unit in RNNs to improve the accuracy.


\section*{Acknowledgment}


This research was partially supported by \textbf{\textit{Framgia Vietnam}}. We are thankful to our colleagues who provided  expertise that greatly assisted the research, although they may not agree with all of the interpretations provided in this paper.


% Can use something like this to put references on a page
% by themselves when using endfloat and the captionsoff option.
\ifCLASSOPTIONcaptionsoff
  \newpage
\fi



% trigger a \newpage just before the given reference
% number - used to balance the columns on the last page
% adjust value as needed - may need to be readjusted if
% the document is modified later
%\IEEEtriggeratref{8}
% The "triggered" command can be changed if desired:
%\IEEEtriggercmd{\enlargethispage{-5in}}

% references section

% can use a bibliography generated by BibTeX as a .bbl file
% BibTeX documentation can be easily obtained at:
% http://www.ctan.org/tex-archive/biblio/bibtex/contrib/doc/
% The IEEEtran BibTeX style support page is at:
% http://www.michaelshell.org/tex/ieeetran/bibtex/
%\bibliographystyle{IEEEtran}
% argument is your BibTeX string definitions and bibliography database(s)
%\bibliography{IEEEabrv,../bib/paper}
%
% <OR> manually copy in the resultant .bbl file
% set second argument of \begin to the number of references
% (used to reserve space for the reference number labels box)
\begin{thebibliography}{1}

\bibitem{1}
Zissman, Marc A., et al. "Automatic dialect identification of extemporaneous conversational, Latin American Spanish speech." Acoustics, Speech, and Signal Processing, 1996. ICASSP-96. Conference Proceedings., 1996 IEEE International Conference on. Vol. 2. IEEE, 1996.
\bibitem{2}
Harrison, Alissa M., et al. "Implementation of an extended recognition network for mispronunciation detection and diagnosis in computer-assisted pronunciation training." International Workshop on Speech and Language Technology in Education. 2009.
\bibitem{3}
Ng, Kenney, and Victor W. Zue. "Phonetic recognition for spoken document retrieval." Acoustics, Speech and Signal Processing, 1998. Proceedings of the 1998 IEEE International Conference on. Vol. 1. IEEE, 1998.
\bibitem{4}
Zeghidour, Neil, et al. "Learning Filterbanks from Raw Speech for Phone Recognition." arXiv preprint arXiv:1711.01161 (2017).
\bibitem{5}
Tóth, László. "Phone recognition with hierarchical convolutional deep maxout networks." EURASIP Journal on Audio, Speech, and Music Processing 2015.1 (2015): 25.
\bibitem{6}
Vaněk, Jan, et al. "A Regularization Post Layer: An Additional Way How to Make Deep Neural Networks Robust." International Conference on Statistical Language and Speech Processing. Springer, Cham, 2017.
\bibitem{7}
Mohamed, Abdel-rahman, George Dahl, and Geoffrey Hinton. "Deep belief networks for phone recognition." Nips workshop on deep learning for speech recognition and related applications. Vol. 1. No. 9. 2009.
\bibitem{8}
Tóth, László. "Combining time-and frequency-domain convolution in convolutional neural network-based phone recognition." Acoustics, Speech and Signal Processing (ICASSP), 2014 IEEE International Conference on. IEEE, 2014.
\bibitem{9}
Graves, Alex, Abdel-rahman Mohamed, and Geoffrey Hinton. "Speech recognition with deep recurrent neural networks." Acoustics, speech and signal processing (icassp), 2013 ieee international conference on. IEEE, 2013.
\bibitem{10}
Garofolo, John S., et al. "DARPA TIMIT acoustic-phonetic continous speech corpus CD-ROM. NIST speech disc 1-1.1." NASA STI/Recon technical report n 93 (1993).
\end{thebibliography}


% biography section
% 
% If you have an EPS/PDF photo (graphicx package needed) extra braces are
% needed around the contents of the optional argument to biography to prevent
% the LaTeX parser from getting confused when it sees the complicated
% \includegraphics command within an optional argument. (You could create
% your own custom macro containing the \includegraphics command to make things
% simpler here.)
%\begin{biography}[{\includegraphics[width=1in,height=1.25in,clip,keepaspectratio]{mshell}}]{Michael Shell}
% or if you just want to reserve a space for a photo:

\begin{IEEEbiography}[{\includegraphics[width=1in,height=1.25in,clip,keepaspectratio]{picture}}]{John Doe}
\blindtext
\end{IEEEbiography}

% You can push biographies down or up by placing
% a \vfill before or after them. The appropriate
% use of \vfill depends on what kind of text is
% on the last page and whether or not the columns
% are being equalized.

%\vfill

% Can be used to pull up biographies so that the bottom of the last one
% is flush with the other column.
%\enlargethispage{-5in}



% that's all folks
\end{document}


